\documentclass{article}
\usepackage[utf8]{inputenc}
\usepackage[spanish]{babel}
\usepackage{graphicx}
\usepackage{amsmath}

\title{\textbf{TuTurismo Neiva: Fomentando la Cultura y el Turismo a Través de la Innovación Tecnológica}}
\author{
    Valentina Silva Garrido \texttt{} \\
    Mariana González Calderón \texttt{} \\
    Maydy Viviana Conde Ladino \texttt{} \\
    Dylan Santiago Narvaez Pinto \texttt{} \\
    Isabela Gutierrez Cordoba \texttt{} \\
    Ivan Andres Murcia Epia \texttt{}
}

\begin{document}

\maketitle

\begin{abstract}
This article aims to present the TuTurismo Neiva application, designed to promote tourism in the city of Neiva through an interactive digital platform. The app allows users to explore, discover and value the tourist sites and monuments of the city, facilitating the integration of local history with modern evaluation and recommendation tools. The methodology used in the development of the app includes agile approaches such as Scrum and XP, as well as graphic tools such as Canva for the creation of the interface. The results obtained show that the app has improved the visibility of lesser-known sites and has favored user interaction with Neiva's cultural heritage. In addition, the platform includes a PQRSFD system, which allows users to express opinions and suggestions, contributing to a process of continuous improvement. In conclusion, TuTurismo Neiva not only fosters cultural knowledge, but also promotes the economic and social development of the city, providing an interactive experience for tourists.
\end{abstract}

\textbf{Palabras Clave:} Aplicación móvil, Turismo, Cultura, Neiva, Innovación, Historia, PQRSFD

\section{Marco Teórico}
El concepto de turismo digital ha ganado relevancia con el avance de la tecnología móvil y la conectividad en línea, permitiendo a los usuarios acceder a información en tiempo real sobre destinos turísticos. Según Dredge (2006), las plataformas digitales fomentan la interacción entre turistas y locales, lo que enriquece la experiencia cultural y promueve la sostenibilidad. De igual manera, los sistemas de retroalimentación como los PQRSFD (Preguntas, Quejas, Reclamos, Sugerencias, Felicitaciones y Denuncias) han sido identificados como cruciales para mejorar la calidad de los servicios turísticos y la satisfacción del cliente (Fitzsimmons, 2014). Además, estudios previos como el de O'Connor (2010) destacan cómo las herramientas tecnológicas pueden transformar la industria del turismo al proporcionar plataformas de fácil acceso para la promoción de destinos.

\section{Metodología}
El desarrollo de TuTurismo Neiva siguió un enfoque ágil utilizando las metodologías Scrum y XP para asegurar un proceso iterativo y flexible en la construcción de la aplicación. Scrum permitió dividir el proyecto en sprints, cada uno enfocado en una funcionalidad específica como la geolocalización de sitios turísticos, la integración del sistema de evaluación de usuarios, y la creación del sistema PQRSFD. Para el diseño visual, se utilizó la plataforma Canva, que facilitó la creación de interfaces intuitivas y visualmente atractivas. Las pruebas de usabilidad se realizaron con un grupo de usuarios locales y turistas, cuyos comentarios fueron fundamentales para ajustar la experiencia y funcionalidad de la app.

\section{Resultados}
TuTurismo Neiva ha logrado reunir una base de datos significativa de sitios turísticos en Neiva, con más de 50 lugares registrados y evaluados por los usuarios. A través del sistema de calificación, los turistas pueden valorar estos sitios, lo que ha ayudado a destacar atracciones menos conocidas. Además, el sistema de PQRSFD ha permitido recopilar información valiosa sobre las experiencias de los usuarios, contribuyendo al mejoramiento continuo de la app. Gráficos y tablas presentadas a continuación muestran el crecimiento en el número de usuarios activos y la distribución geográfica de los sitios más visitados. (Gráficos y tablas aquí: imágenes con datos de usuarios, calificaciones de sitios, y distribución geográfica de visitas.)

\section{Discusión}
Los resultados obtenidos en este estudio sugieren que TuTurismo Neiva ha
logrado cumplir con su objetivo de incrementar la visibilidad y accesibilidad de los
sitios turísticos de la ciudad. Sin embargo, se ha observado que los usuarios aún
prefieren los sitios más conocidos, lo que resalta la necesidad de mejorar las
estrategias de marketing y promoción digital de los lugares menos visitados.
Comparado con estudios previos en plataformas de turismo digital, la integración
de un sistema de retroalimentación como el PQRSFD en la app ha demostrado ser
una práctica efectiva para fomentar la interacción con los usuarios y mejorar los
servicios. Las limitaciones incluyen la falta de recursos para la promoción más
amplia de la app a nivel nacional y la necesidad de realizar más pruebas a gran
escala.

\section{Conclusiones:}
En conclusión, TuTurismo Neiva ha logrado crear una plataforma eficaz para
promover el turismo en Neiva, mejorando la interacción de los turistas con la
cultura local. La implementación de la metodología ágil, combinada con el uso de
herramientas de diseño intuitivas, ha permitido la creación de una aplicación
funcional y de fácil acceso. Se recomienda expandir la app a otras ciudades y
realizar una campaña de marketing digital más agresiva para aumentar el número
de usuarios y la visibilidad de los sitios menos conocidos. Además, se sugiere
incorporar más funcionalidades, como itinerarios personalizados y una plataforma
para que los usuarios creen sus propios recorridos turísticos.

\end{document}
